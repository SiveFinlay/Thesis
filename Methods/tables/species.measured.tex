%Species.measured table
%To get the number measured for each Family, I counted the number of unique species in my Tb_Taxonomy Access table for that family
%So it's the number of species measured overall but that doesn't necessarily mean that I have the same number in every data set
%I didn't count the _sp. specimens as separate species

%I took out the reference to the IUCN because it seems to only list species that are threatened/endangered rather than a full list of all species in a family

\begin{tabular}{p{3.4cm}p{3cm}p{2cm}p{2cm}p{2cm}}

\hline
\textbf{Order} & \textbf{Family} & \textbf{Measured species} & \textbf{Total species} & \textbf{Coverage} \\
\hline
%----------------------------------------------------
Afrosoricida & Tenrecidae & 31 & 34 & 91\% \\
%-------------------------------------------------
Afrosoricida & Chrysochloridae & 12 & 21 & 57\% \\
%----------------------------------------------------
Erinaceomorpha & Erinaceidae & 16 & 24 & 67\% \\
%----------------------------------------------------
Soricomorpha & Soricidae & 22 & 376 & 5.8\% \\
%----------------------------------------------------
Soricomorpha & Solenodontidae & 2 & 4 & 50\% \\
%----------------------------------------------------
Soricomorpha & Talpidae & 15 & 39 & 38\% \\
%----------------------------------------------------
Notoryctemorphia & Notoryctidae & 1 & 2 & 50\% \\
%-----------------------------------------------
\hline

\end{tabular}

%*There are now 34 recognised tenrec species \citep{Olson2013}