%Species.measured table
%To get the number measured for each Family, I counted the number of unique species in my Tb_Taxonomy Access table for that family
%So it's the number of species measured overall but that doesn't necessarily mean that I have the same number in every data set
%I didn't count the _sp. specimens as separate species

%I took out the reference to the IUCN because it seems to only list species that are threatened/endangered rather than a full list of all species in a family

\begin{tabular}{llcc}

\hline
\textbf{Order} & \textbf{Family} & \textbf{Species measured} & \textbf{Species in MSW} \\
\hline
%----------------------------------------------------
Afrosoricida & Tenrecidae & 31 & 30 \\
%-------------------------------------------------
Afrosoricida & Chrysochloridae & 12 & 21\\
%----------------------------------------------------
Erinaceomorpha & Erinaceidae & 16 & 24 \\
%----------------------------------------------------
Soricomorpha & Soricidae & 22 & 376 \\
%----------------------------------------------------
Soricomorpha & Solenodontidae & 2 & 4 \\
%----------------------------------------------------
Soricomorpha & Talpidae & 15 & 39 \\
%----------------------------------------------------
Notoryctemorphia & Notoryctidae & 1 & 2 \\
%-----------------------------------------------
\hline
\end{tabular}

%*There are now 34 recognised tenrec species \citep{Olson2013}