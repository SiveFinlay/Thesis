\chapter*{Summary}
\chaptermark{summary}
\addcontentsline{toc}{chapter}{Summary}

	Patterns of morphological diversity (the variety of physical form) have important implications for our understanding of species ecology and evolution. Some clades such as cichlid fish and anole lizards are known for their high morphological diversity. Studying these groups can provide insights into the processes that led to the evolution of this exceptional diversity. 

	Morphological diversity is often studied qualitatively. However, to truly understand the evolution of diversity, it is important to take a quantitative approach instead of relying on subjective, qualitative assessments. Here, I present a quantitative analysis of morphological diversity in a Family of small mammals, the tenrecs (Afrosoricida, Tenrecidae). 
	
	Tenrecs are often cited as an example of an exceptionally morphologically diverse group. However, this assumption has not been tested quantitatively. In this thesis, I use geometric morphometric analyses of skull shape to test whether tenrecs are more morphologically diverse than their closest relatives, the golden moles (Afrosoricida, Chrysochloridae). Tenrecs occupy a wider range of ecological niches than golden moles so I predicted that they would be more morphologically diverse. 
	
	The results of my analyses are mixed. When all species were included, tenrec skulls were only more morphologically diverse than golden moles in lateral view. The morphologically similar, species-rich \textit{Microgale} tenrec Genus appears to mask higher morphological diversity in the rest of the Family: when the over-representation of this Genus is taken into account then tenrecs are more morphologically diverse than golden moles.
	Overall, my results appear to support the designation of tenrecs as an exceptionally diverse group however they also highlight the importance of identifying the level at which diversity should be measured (i.e. whether diversity includes all species individually or representative samples of morphologically similar groups).
	
	These results demonstrate the importance of using quantitative methods to test qualitative assumptions about patterns of morphological diversity. In addition, the extensive morphological data set I collected represents a significant resource for future research.
	
