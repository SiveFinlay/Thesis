

\chapter{Data appendix}
\label{appendix}

%Needs a lead in before getting to the linear measurements

\section{\normalfont{Linear measurements}} 
\label{sect:measurements} %Label the section so I can refer to it in the error checking later on

	Using 15 mm digital calipers (Mitutoyo Absolute digimatic calipers), I took five measurements from each mandible (table \ref{tab:mands.measurements}), 15 from each skull (table \ref{tab:sk.measurements}) and 19 from each set of limbs (table \ref{tab:limb.measurements}). My choice of  measurements was based on three main criteria; 1) their relevance to biological and ecological traits such as diet specialisation and locomotory adaptation; 2) their usefulness for assessing the overall shape and size of the specimen; and 3) the ease with which they could be repeated both within and among specimens from different species. 
	%Figures x-x depict the linear measurements of skulls and figures xx show the limb measurements.
	%Maybe I could get away with no pictures?
	%It would be tricky to show all of the measurements on single images

	% NC: Depends. If you ever want people to be able to use this data then pictures are necessary aren't they? However, you could just cut all the linear measurement stuff as you don't use it at all. I'd be tempted to do this because your thesis is already huge! You could pop this all into supplementary.

	I took each linear measurement three times, cycling through all 20 skull or 19 limb measurements then repeating the cycle to avoid measuring the same variable twice in a row. Small measurements (<2 mm) are particularly prone to high error rates \citep{Cardini2008}. Therefore, I took five separate replicates of some of the measurements which were often less than 2 mm and consequently most prone to errors (marked with * in tables \ref{tab:sk.measurements} and \ref{tab:limb.measurements}). These included four of the skull measurements (PWa, IncisorH, IFD and IFcanal, table \ref{tab:mands.measurements}) and five of the limb measurements (FemD, TibD, HumD, UlnD, RadD, table \ref{tab:limb.measurements}). 
	Five replicates should give a more reliable median value because even if there are one or two outlying measurements there should be at least three replicates which are in close agreement \citep{Cooper2009}.
	
	%Could put in an explanation of why there are extra measurements for golden mole limbs and why I used minimum diameter of the bones - loading capacity

%******************************
%The tables are very long so maybe I should stick them into an appendix or give shorter descriptions of the measurements?

% NC: No leave them here or the rest of the methods get hard to understand. 3 pages is fine. Also can you please sort out the widths of the last column. I showed you how to do it before.

% Mandible  measurements

\begin{table}[!htbp]
	\caption[Mandible measurements]
			{Measurement abbreviations and descriptions for the mandibles, all taken from the labial (outer) side of the right jaw unless that side was broken or missing. All measurements were repeated three times except for those marked with * which were measured five times.}
	\input{Methods/tables/mands.measurements}
	\label{tab:mands.measurements}
\end{table}

%Skull measurements
\begin{table}[!htbp]
	\caption[Skull measurements]
			{Measurement abbreviations and descriptions for the skulls. All measurements were repeated three times except for those marked with * which were measured five times.}% add to this caption
	\input{Methods/tables/skulls.measurements}
	\label{tab:sk.measurements}
\end{table}



% Limb measurements
\begin{table}[!htbp]
	\caption[Limb measurements]
		{Measurement abbreviations and descriptions for the limbs. All measurements were repeated three times except for those marked with * which were measured five times.} % add to this caption
	\input{Methods/tables/limbs.measurements}
	\label{tab:limb.measurements}
\end{table}


%****************************
\subsection{\normalfont{Error checking for linear measurements}}

	%Maybe this isn't relevant anymore since I'm not actually doing anything with the linear measurement data?
	%I've kept it in for the moment as a reference incase anyone wants to use the data in the future
	% NC: Yeah I'd pop this into the supp material
	
	As mentioned above (section \ref{sect:measurements}), I took three replicate measurements of most of my variables and five replicates of other, smaller variables. 
	Some morphometric studies take replicate measurements of a trait and use the average value for further analyses (REFS?). Rather than taking the mean of each of three (or five) measures, I used the median as it is less likely to be skewed by outliers and gives a more accurate representation of the true value of the trait (REFS?).
	
	%Come back to here for references
	
	Before extracting the median values, I followed the protocol for assessing measurement error outlined by \citep{Cooper2009}. This method assesses whether there is a reasonable correlation among the replicate measurements of the same variable. The error checking criteria are based on two calculations; the coefficient of variation and the percentage spread.
	
	I calculated the coefficient of variation (standard deviation/mean*100) for each measurement. This value estimates the extent to which replicate measurements deviate from the mean. When the coefficient of variation was less than 5\%, I accepted the median value as an accurate measurement of the size of the structure. 
	If the coefficient of variation was greater than 5\%, indicative of a low agreement between replicate measurements, I measured the percentage spread of the data. For variables measured three times, I calculated percentage spread as [(minimum difference between neighbouring measurements)/ (range of measured values)*100].
	For variables that I measured five times, the differences between neighbouring values were calculated and labelled from smallest to largest as a, b, c, and d with the range of the measured values designated as e \citep{Cooper2009}. For these variables, I calculated percentage spread as [(a/e + b/e + c/e)*100]. 
	Small percentage spread values indicate close agreement between repeated measurements. When percentage spread approaches 50\% the data are evenly spread out and therefore there is no way of knowing whether the median value is an accurate measurement of the trait \citep{Cooper2009}. I chose to use to use 25\% as a cut off point for accepting the accuracy of measured traits.

	I used these error checking criteria to assess the accuracy of my repeated measurements of both skulls and limbs. 

%I've taken this out for now since I'm not actually using any of the linear measurement data
	%Of the 20 measurements for xxx skulls, there were xx variables belonging to xx skulls which had coefficient of variation > 5\% and percentage spread >25 \%. My final skull data set included xx replicates of xx variables from xx specimens comprising xx species.

	%Of the 19 measurements for xxx limbs, there were xx variables belonging to xx specimens which had coefficient of variation > 5\% and percentage spread >25 \%. My final limb data set included xx replicates of xx variables from xx specimens comprising xx species.

%--------------------------------------
\section{Photographing specimens}
%Separate out the limbs and the skins

	Initially, I tried to take pictures of the limbs in similar orientations to the skulls (dorsal, ventral and lateral). However, there was considerable variation in how the limbs were preserved. For example, some limbs were still articulated while others had fragmented bones. It therefore proved impossible to place the limb bones in consistent orientations that would be comparable across species. Similarly, the small size of some limbs, combined with the frequently incomplete nature of postcranial museum collections, made landmark-based morphometric analyses of any limb pictures impractical. Therefore, I photographed the fore- and hind-limb bones in outer (the side facing away from the rest of the body) and inner (the side facing in towards the centre of the body) views for reference purposes only.

	As I was limited by the maximum camera height available on the copy stands, most skins were too large to be photographed with the 100mm macro lens. Therefore, I used an EFS 18-55mm lens to take pictures of the skins. I photographed skins in the same three orientations as the skulls; dorsal (the upper surface of the animal), ventral (the belly side of the skin) and lateral (right flank of the animal with the skin held in position using bean bags). The dorsal and ventral views give very approximate estimates of the overall body shape of the animal. The lateral views are less biologically relevant since the taxidermic process is unlikely to produce specimens which represent the true body height of the animal.

%--------------------------------------------
