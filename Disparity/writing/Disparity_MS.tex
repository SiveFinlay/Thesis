%Start: 22/04/14
%Last edited on 17/09
	%Complete revision after committee meeting and NERD club
	%More straightforward story, clearer results with a more simple measure of diversity, took out the mandibles completely
	%Now aiming for the Journal of Mammalogy (needs very different formatting)


% Preamble
\documentclass[12pt,a4paper]{article}
\usepackage{enumerate} 	% put in numbers or bullet points
\usepackage{setspace}	% line spacing					
\usepackage{authblk}	% For author affiliations
\usepackage{graphicx} 	% For adding pictures

\usepackage[nomarkers]{endfloat} %Figures and tables at the end of the document
\usepackage{pdflscape}	% for landscape pages
\usepackage{mathtools}	% For equations etc.
\usepackage[osf]{mathpazo} % palatino font package
\usepackage{fixltx2e}	% includes subscripts
\usepackage{ms}     	% load the manuscript style template

%\usepackage{float}		% Use these options to have figures in specific places in the text
%\floatstyle{plaintop} 	% Force table captions to go above the table

\setcounter{secnumdepth}{0} % removes numbers from section headings
\raggedright 			% justify the text on the left only
\pagenumbering{arabic}	% Page numbers

%\onehalfspacing 		%1.5 line spacing - use the below instead in case you need double spacing for some journals
%\linespread{1.6} 		% this is 1.5 spacing, double line spacing is 1.6 - unrecognised command (maybe due to the setspace package?)
\usepackage[round]{natbib} % author-year citations in round brackets
%-----------------------------------------
%Title page
%----------------------------------------
\title{Morphological diversity of tenrec (Afrosoricida, Tenrecidae) skulls compared to their closest relatives, the golden moles (Afrosoricida, Chrysochloridae)} 
% I want to come up with a better title

\author{Sive Finlay$^{1,2,*}$ and Natalie Cooper$^{1,2}$}
\affiliation{\noindent{\footnotesize
$^1$ School of Natural Sciences, Trinity College Dublin, Dublin 2, Ireland.\\ 
$^2$ Trinity Centre for Biodiversity Research, Trinity College Dublin, Dublin 2, Ireland.\\
$^*$Corresponding author: sfinlay@tcd.ie; Zoology Building, Trinity College Dublin, Dublin 2, Ireland.\\ Fax: +353 1 6778094; Tel: +353 1 896 2571.\\}}
\date{}	% To give blank date

\runninghead{Cranial morphological diversity in tenrecs } %Need to fix this when I have a proper title

\keywords{geometric morphometrics, golden moles, morphological diversity, tenrecs}
%---------------------------------------------------------------
% Start of document
\begin{document}

\modulolinenumbers[1] 	% Line numbering on every line

\mstitlepage			% Instead of \maketitle you can use the nice template to get it looking like a manuscript
\parindent=1.5em		% Changes paragraph indenting so it's not so big
\addtolength{\parskip}{.3em} % Changes spacing between sections so it's smaller
%---------------------------------------------------
\begin{abstract} 

% Abstract need to be no more than 5 % of the rest of the text
	% Main message is the importance of quantitative vs. qualitative

	Morphologically diverse groups have long attracted the interest of biologists. Many studies now recognise the importance of quantifying patterns of morphological diversity to gain new insights into evolutionary patterns. Tenrecs (Afrosoricida, Tenrecidae) are a family of small mammals which is often cited as an example of an exceptionally morphologically diverse group. However, this assumption has not been tested. Here we use geometric morphometric analyses of skull shape to test whether tenrecs are more morphologically diverse than their closest relatives, the golden moles (Afrosoricida, Chrysochloridae). Contrary to our expectations, we find that tenrec skulls are only more morphologically diverse than golden moles when measured in lateral view. Furthermore, the similarities among the species-rich \textit{Microgale} tenrec Genus appear to mask higher morphological diversity in the rest of the Family. Our results reveal new insights into the morphological diversity of tenrecs and highlight the importance of using quantitative methods to test qualitative assumptions about patterns of morphological diversity.
	
\end{abstract}

\newpage
%-------------------------------------------------------
\section{Introduction} 

%1) Morphological diversity
	Morphological diversity has long attracted the attention of biologists. There are many famous examples of interesting morphological variation including beak morphologies in Darwin's finches, body and limb morphologies in Caribbean \textit{Anolis} lizards and pharyngeal jaw diversity in cichlid fish \citep{Gavrilets2009}. %Or maybe find separate references for each one?
	Apart from a few examples \citep[e.g.][]{Goswami2011, Ruta2013, Brusatte2008}, it is still common to study morphological diversity from a qualitative rather than quantitative perspective. However, it is important to quantify morphological diversity because it has implications for studies of adaptive radiations \citep{Losos2010}, convergent evolution \citep[e.g.][]{Muschick2012, Harmon2005} and our understanding of biodiversity \citep{Roy1997}.
	

%2) Tenrecs

	Tenrecs (Afrosoricida, Tenrecidae) are an example of a morphologically diverse group \citep{Soarimalala2011, Olson2003}. The Family contains 34 species, 31 of which are endemic to Madagascar \citep{Olson2013}. Body sizes of tenrecs span three orders of magnitude (2.5 to $>$ 2,000g) which is a greater range than all other Families, and most Orders, of living mammals \citep{Olson2003}. Within this vast size range there are tenrecs which convergently resemble shrews (\textit{Microgale} tenrecs), moles (\textit{Oryzorictes} tenrecs) and hedgehogs (\textit{Echinops} and \textit{Setifer} tenrecs) \citep{Eisenberg1969} even though they are not closely related to these species \citep{Stanhope1998}. There are some qualitative similarities in the morphology of some tenrecs' limbs compared to other species \citep{Salton2009}. However, the apparent morphological diversity of tenrecs has not been quantified.

%3) Difficult to measure (in general and back to tenrecs)

	Morphological diversity is difficult to quantify. Studies are inevitably constrained to measure the diversity of specific traits rather than overall morphologies \citep{Roy1997}. Different trait axes (such as cranial compared to limb morphologies) may yield different patterns of morphological diversity \citep{Foth2012}.
	Furthermore, linear measurements of morphological traits can restrict our understanding of overall morphological variation \citep{Rohlf1993}. However, geometric morphometric approaches \citep{Rohlf1993, Adams2013} provide more detailed insights into morphological variation.
	 
%4) Summary of findings
	Here we present the first quantitative investigation of morphological diversity in tenrecs. We use geometric morphometric approaches to compare cranial morphological diversity in tenrecs to their sister taxa, the golden moles (Afrosoricida, Chrysochloridae). We compare skull morphologies in three different views: dorsal, ventral and lateral. 
	Tenrecs inhabit a wider variety of ecological niches \citep{Soarimalala2011} than golden moles \citep{Bronner1995} so we expected tenrecs to be more morphologically diverse than their closest relatives. However, we only find a significant difference in the morphological diversity of skulls in lateral view, not dorsal or ventral. In contrast, when we restricted our data to include a subsample of the morphologically similar \textit{Microgale} tenrec Genus, we found that tenrecs were more morphologically diverse than golden moles in all three analyses.
	Our results highlight the importance of using quantitative methods to test assumptions about patterns of morphological diversity.
	%Needs a better last line here

%-------------------------------------------------------------
\section{Materials and Methods}

	Our methods for measuring cranial morphological diversity involved several steps of data collection, processing and analysis. For clarity,  figure \ref{fig:flow} summarises all of these steps which are described in detail below.   
	
	%*************************************************
	%Methods flowchart
	%**************************************
		%landmarks diagram
		\begin{figure}
		\centering
		\includegraphics[width=1\linewidth]{figures/Methods_flowchart.png}
		
		\caption[Flowchart diagram of data collection and analysis]
			{Summary of the main steps in our data collection, processing and analysis protocol. Note that skulls were photographed in three views and then the ensuing analyses were repeated separately for each view. The dashed arrows refer to the analyses we repeated while including only a subset of \textit{Microgale} tenrecs.}
		
		\label{fig:flow}
		\end{figure}
	
	
	%************************************************** 

\subsection{Morphological data collection} 
	
	One of us (SF) photographed cranial specimens of tenrecs and golden moles at the Natural History Museum London (BMNH), the Smithsonian Institute Natural History Museum (SI), the American Museum of Natural History (AMNH), Harvard's Museum of Comparative Zoology (MCZ) and the Field Museum of Natural History, Chicago (FMNH). We photographed the specimens with a Canon EOS 650D camera fitted with an EF 100mm f/2.8 Macro USM lens using a standardised procedure to minimise potential error (see supplementary material for details). 
		%SF: Remember my pictures are labelled NHML so I can just put a note with the figshare folder

	We collected pictures of the skulls in dorsal, ventral and lateral views (right side of the skull). A full list of museum accession numbers and details on how to access the images can be found in the supplementary material.
	
	
	%AMNH and SI pictures are on figshare but MCZ and FMNH are more tricky about copyright so I haven't put those pictures up
	% NC: May be worth linking to figshare here rather than just in suppl
	%Each of the picture sets have a different doi reference so I thought it might look clumsy to stick them in here. I've referred to each of them separately in the supplementary.

	In total we collected pictures from 182 skulls in dorsal view (148 tenrecs and 34 golden moles), 173 skulls in ventral view (141 tenrecs and 32 golden moles) and 171 skulls in lateral view (140 tenrecs and 31 golden moles) representing 31 species of tenrec (out of the total 34 in the family \citep{Olson2013}) and 12 species of golden moles (out of a total of 21 in the family \citep{Asher2010}). We used the taxonomy of Wilson and Reeder \citeyearpar{Wilson2005} supplemented with more recent sources \citep{Olson2013} to identify our specimens. 
	

	We used a combination of landmarks (type 2 and type 3, \citep{Zelditch2012}) and semilandmarks to characterise the shapes of our specimens. Figure \ref{fig:skulls_landmarks} shows our landmarks (points) and semilandmarks (outline curves) for the skulls in each of the three views. Corresponding definitions of each of the landmarks can be found in the supplementary material.
	
	%semilandmark or semi-landmark? The Latter might improve readability
	%SF: it's usually semilandmark in other papers

	We used the TPS software series \citep{SUNY2009} to process and landmark the pictures (Fig. \ref{fig:flow}). We digitised all landmarks and semilandmarks in tpsDIG, version 2.17 \citep{Rohlf2013}. We re-sampled the outlines to the minimum number of evenly spaced semilandmark points required to represent each outline accurately \citep[][details in supplementary material]{MacLeod2013}. We used TPSUtil \citep{Rohlf2012} to create "sliders" files that defined which points in our TPS files should be treated as semilandmarks \citep{Zelditch2012}. We conducted all subsequent analyses in R version 3.0.2 \citep[][Fig. \ref{fig:flow}]{Team2014}. 
	
	We used the gpagen function in the geomorph package \citep{Adams2013} to run a general Procrustes alignment \citep{Rohlf1993} of the landmark coordinates while sliding the semilandmarks by minimising Procrustes distance \citep{Bookstein1997}. We used these Procrustes-aligned coordinates of all species to calculate average shape values for each species (n = 43) which we then used for a principal components analysis (PCA) with the plotTangentSpace function \citep{Adams2013}. 
	
	%Phylogenetic PCA of geometric morphometrics data doesn't affect distance-based morphological comparisons (identical results to normal PCA, Polly 2013). So I could put this in as a justification for using a normal rather than phylogenetic PCA approach?


%*************************************************
%Combined picture of three skull views
%**************************************
	%landmarks diagram
	\begin{figure}
	\centering
	\includegraphics[width=1\linewidth]{figures/skdors+skvent+sklat_BW.png}
	
	\caption[Diagram of the landmarks and curves for the skulls in dorsal ventral and lateral views]
		{Landmarks (numbered points) and curves (black lines) used to capture the morphological shape of skulls in dorsal, ventral and lateral views respectively. Curves were re-sampled to the same number of evenly-spaced points. See supplementary material for descriptions of the curves and landmarks. The specimens belong to two different \textit{Potamogale velox} (Tenrecidae) skulls: accession number AMNH 51327 (dorsal) and BMNH 1934.6.16.2 (ventral and lateral)}
	
	\label{fig:skulls_landmarks}
	\end{figure}


%************************************************** 

	
%-------------------------------------------------------	
\subsection{Calculating morphological diversity}

	We calculated morphological diversity using the results of our principal components analyses. We selected the principal components axes which accounted for 95\% of the cumulative variation for each of our three skull analyses. These axes represent the dimensions of our morphospace \citep{Polly2013}. We used the scores from the PC axes to compare cranial morphologies in two ways (Fig. \ref{fig:flow}).
	
	First, we used non parametric MANOVAs \citep{Anderson2001} to test whether tenrecs and golden moles occupied significantly different positions within our cranial morphospaces \citep[e.g][]{Serb2011, Ruta2013}.
	
	Secondly, we compared morphological diversity within tenrecs to the diversity within golden moles. We calculated the morphological diversity of each Family as the mean Euclidean distance between every species and the centroid for that Family. If tenrecs are more morphologically diverse, then they should be more spread-out within our cranial morphospaces. We used a t-test to assess whether there was any significant difference in the morphological diversity (spread in morphospace) of tenrecs and golden moles.
	
	Our groups have unequal sample sizes (31 tenrec species compared to 12 golden mole species). Morphological diversity is usually decoupled from taxonomic diversity \citep[e.g.][]{Ruta2013, Hopkins2013} so larger groups are not necessarily more morphologically diverse. However, comparing morphological diversity in tenrecs to the diversity of a smaller Family could still bias our results. To account for this, we used pairwise permutation tests. Our null hypothesis was that there is no difference in morphological diversity between tenrecs and golden moles. If this were true, then the group identity of each species would be arbitrary: if you randomly assign the species as being either a tenrec or golden moles and then re-calculate morphological diversity there would still be no difference in the diversity of the two groups. 
	
	We assigned Family identities at random to each species and calculated the differences in morphological diversity (mean Euclidean distances to the Family's centroid) for the new groupings. We repeated these permutations 1000 times to generate a null distribution of the expected differences in morphological diversity between a group that has 31 members (tenrecs) compared to one which has 12 members (golden moles). Finally, we compared our observed (true) measures of the differences in morphological diversity between the two Families to our permuted distributions to test whether there were significant differences after taking sample size into account.
	
	The majority of tenrec species (19 out of 31 in our dataset) are members of the \textit{Microgale} (shrew-like) Genus which is notable for its relatively low morphological diversity \citep{Soarimalala2011, Jenkins2003}. Therefore, the strong similarities among these species may mask signals of higher morphological diversity among other tenrecs. 
	To test this idea, we created a subset of our tenrec data which included just five of the \textit{Microgale} species. Each species represents one of the five sub-divisions of \textit{Microgale} outlined by Soarimalala and Goodman \citeyearpar{Soarimalala2011}: four categories of body size (small, small-medium, medium, large) and long-tailed species. We compared the morphological diversity of this subset of tenrecs (n=19: 5 \textit{Microgale} with the 12 other tenrec species) to the morphological diversity within the 12 species of golden moles. We used the same morphological diversity comparisons and permutation tests to account for differences in sample size on this reduced data set (Fig. \ref{fig:flow}).
	 


% NC: Now this is shortened I'd stick the rarefaction in here too
	% SF: I took out rarefaction because Steve Wang and Steve Brusatte both advised that it was not the most appropriate test to use. The permutation method also takes sample size into account so I thought that removed the need for a rarefaction analysis aswell?
%-----------------------------------------------------------

\section{Results}
 
	%New axes numbers (select axes for 95 % of the variation, not one extra)
		%Full data: skdors =6, skvent=7, sklat =7
		%Microgale subset: skdors=6, skvent=6, sklat=6
	Figure \ref{fig:sklatPCA} depicts the morphospace plot derived from our principal components analysis of average Procrustes-superimposed shape coordinates for skulls in lateral view. Similar plots for our analyses of skulls in dorsal and ventral views can be found in the supplementary material.
	To compare morphological diversity in the two families, we used the principal components axes which accounted for 95\% of the cumulative variation in each of our skull analyses: dorsal (n=6 axes), ventral (n=7 axes) and lateral (n=7 axes). 
	
	First, we compared the position of each Family within the morphospace plots. Tenrecs and golden moles occupy significantly different positions in the dorsal 	(npMANOVA, F \textsubscript{1,42} = 68.13, R$^2$ = 0.62, p=0.001 ), ventral (npMANOVA, F \textsubscript{1,42} = 103.33, R$^2$ = 0.72 , p=0.001 ) and lateral (npMANOVA, F \textsubscript{1,42} = 76.7, R$^2$=0.652, p=0.001 ) skull morphospaces,  indicating that the Families have very different, non-overlapping cranial morphologies. 
	
	%Numbers are from the npMANOVA based on PC axes within my diversity_twofamily_cent_dist script

	Secondly, we compared the morphological diversity within each Family. Based on our measures of mean Euclidean distances to the Family's centroid, tenrec skulls are more morphologically diverse than golden mole skulls when they're measured in lateral view but not in dorsal or ventral view (table \ref{tab:diversity}). In contrast, when we compared morphological diversity within the sub-sample of 19 tenrecs (including just 5 \textit{Microgale} species) to the 12 golden mole species, we found that tenrecs had significantly higher morphological diversity than golden moles in all analyses (table \ref{tab:diversity}).

	Our pairwise permutation tests for each analysis confirmed that (lack of) differences in morphological diversity were not artefacts of differences in sample size (see supplementary material).
		%Add the new permutation results to the supplementary
%************************************
%Results tables and figures
%Reduced it down to one table and one figure

%This table is very squashed so I could break it up into two?
	\begin{table}[h]			
	\caption[Comparison of morphological diversity in tenrecs and golden moles.]
	\centering{Morphological diveristy in tenrecs and golden moles for each of the three analyses (skulls in dorsal, ventral and lateral view). We measured morphological diversity as the mean Euclidean distance between each species and the centroid for their Family. We compared the morphological diversity of 12 species of golden mole to a) all 31 species of tenrec (left) and b) 19 species of tenrec (right) which included just 5 species of \textit{Microgale} tenrec. Significant differences (p values from t-test comparisons) are highlighted in bold.}
	%Diversity based on centroid distances results summary
%All tenrecs and golden moles
%Morphological diversity based on comparing the mean Euclidean distances to each family's centroid
%NB: Journal of Mammalogy asks for the degrees of freedom that accompany the t test but I'm getting different degrees of freedom in each summary and I don't know why

%Re-ordered the table so that everything would fit in better


\resizebox{\columnwidth}{!}{
%Scales down the table to fit within the column width
	% If this is too small then I'll probably need to break the table into two
\begin{tabular}{c l c c c c}		
\hline
Tenrec species & Analysis & Tenrecs  & Golden moles & t & p \\
%--------------------------------
& & (mean$\pm$ s.e) & (mean$\pm$ s.e) & &\\
\hline
%\multicolumn{1}{l}
%---------------------------
 31 & Skulls dorsal & \multicolumn{1}{l}{0.036 $\pm$ 0.0029} & 0.029 $\pm$ 0.0032 & -1.63 & 0.11 \\
%--------------------------------------
 & Skulls ventral & \multicolumn{1}{l}{0.048 $\pm$0.0034} & 0.044 $\pm$ 0.0041 & -0.676 & 0.51\\
%-----------------------------------------
 & Skulls lateral & 0.044 $\pm$ 0.0041 & 0.032 $\pm$ 0.0037 & -2.16 & \textbf{0.04}\\
%----------------------------------------
 & Mandibles & 0.049 $\pm$ 0.0044 & 0.067 $\pm$ 0.0054 & 2.62 & \textbf{0.014}\\
%--------------------------
\hline
%-----------------------------------------
17 & Skulls dorsal & 0.044 $\pm$ 0.0025 & \multicolumn{1}{l}{0.029 $\pm$0.003} & -3.62 & \textbf{0.001}\\
%---------------------------------
 & Skulls ventral & \multicolumn{1}{l}{0.054 $\pm$ 0.004} & \multicolumn{1}{l}{0.042 $\pm$ 0.004} & -2.23 & \textbf{0.04}\\
%-------------------------------------
 & Skulls lateral &  \multicolumn{1}{l}{0.054 $\pm$ 0.005} & 0.031 $\pm$ 0.0037 & -3.47 & \textbf{0.002} \\
%--------------------
 & Mandibles & 0.055 $\pm$ 0.0049 & \multicolumn{1}{l}{0.062 $\pm$ 0.005} & 1.003 & 0.325 \\
%--------------------

\hline
\end{tabular}
} 
	\label{tab:diversity}  
	\end{table}

		
%PCA figure: Just sklat as an example
	%From the diversity_twofamily_cent_dist script
	\begin{figure}[H]
	\centering
	\includegraphics[width=1\linewidth]{figures/sklat_PCA_allspecies_BW.png}
	\caption[Morphospace (principal components) plot of morphological diversity in lateral views of tenrec and golden mole skulls.]
		{Principal components plot of the morphospace occupied by tenrecs (triangles, n = 31 species) and golden moles (circles, n = 12) for the skulls in lateral view. Each point represents the average skull shape of an individual species. Axes are PC1 and PC2 of the average scores from a PCA analysis of mean Procrustes shape coordinates for each species.}
	\label{fig:sklatPCA}
	\end{figure}
%**********************************************


\section{Discussion} 

%1) Results: strange and why 
	Our results highlight the importance of using quantitative methods to test qualitative assumptions about patterns of morphological diversity. Tenrecs are often cited as an example of a group with high morphological diversity \citep{Olson2013, Soarimalala2011, Eisenberg1969} and we expected them to be more morphologically diverse than their closest relatives. However, tenrecs were only more morphologically diverse than golden moles in just one (lateral view) of our three skull analyses (table\ref{tab:diversity}). Furthermore, the morphologically similar \textit{Microgale} Genus seems to mask high morphological diversity in the rest of the tenrec Family: reducing our data to include a sub-sample of this Genus revealed that the remaining tenrecs were significantly more morphologically diverse than golden moles (table \ref{tab:diversity}). 

	
%2) Why are lateral skulls different
	In our full analyses, tenrecs only had higher morphological diversity than golden moles when the skulls were measured in lateral view. This is most likely due to our choice of landmarks. The two outline curves in lateral view (Fig. \ref{fig:skulls_landmarks}) emphasise morphological variation in the back and top of the skulls, indicating that tenrecs are more morphologically diverse than golden moles in their three dimensional height. 
	These lateral aspects of the skull morphology could not be included in the dorsal and ventral analyses. In contrast, our landmarks in the dorsal, and particularly ventral, views focus on morphological variation in the overall outline shape of the skull and palate (Fig. \ref{fig:skulls_landmarks}). The result that tenrecs are no more diverse than golden moles in these areas makes intuitive sense: most tenrecs have broad, non-specialised diets \citep{Olson2013} so there is no obvious functional reason why they should have significantly diverse palate morphologies.
	Therefore, comparing the morphologies in three separate views allowed us to identify the more morphologically variable skull regions. 
	
	%Another sentence here about the importance of taking different 2D aspects for geometric morphometrics of a 3D object?
	
	
%3) Why does sub-sampling the Microgale make a difference?
	Measures of morphological variation are sensitive to the sampling used. If a particular morphotype is over-represented then the similarities among those species will reduce the overall morphological variation within the group \citep{Foote1991}. This appears to be the case for our data: it is only when we included a sub-sample of \textit{Microgale} tenrecs that we found overall higher morphological diversity in tenrecs compared to golden moles (table \ref{tab:diversity}). These results indicate that the overall morphological diversity within tenrecs is not as large as is often assumed \citep[e.g.][]{Eisenberg1969, Olson2013} because the majority of the Family are members of a single, morphologically similar Genus.

%4) Difficulties and caveats: body size, trait choice, morphological proxy

	Of course our results are based on a single morphological axis; the diversity of skull shape. It is difficult to quantify overall morphological diversity because any study is inevitably constrained by its choice of specific traits \citep{Roy1997}. Many other studies have also used skulls to study morphological variation within species \citep{Blagojevic2011, Bornholdt2008}, to delineate species boundaries within a clade \citep[e.g.][]{Panchetti2008} or for cross-taxonomic comparative studies of morphological (dis)similarities \citep[e.g.][]{Ruta2013, Goswami2011, Wroe2007}. 
	However, variation in skull shape is only one aspect of overall morphology. Quantifying variation in other morphological traits could yield different patterns. Therefore future work should extend our approach beyond just skulls to gain a more complete understanding of the overall morphological diversity of tenrecs and golden moles. 
		
%5) Conclusions
	We have presented the first quantitative investigation of morphological diversity in tenrecs. We found that tenrec skulls are more morphologically diverse than their closest relatives but only in some aspects of their morphology. Furthermore, our results indicate that the similarities among the species-rich \textit{Microgale} tenrecs seem to mask signals of higher morphological diversity among the rest of the Family. Of course our results are restricted to just one axis of morphological variation and further analysis of other traits is required. However, our results represent a significant step towards a more accurate, quantitative understanding of otherwise subjective assessments of patterns of morphological diversity. 


%---------------------------------------------------	
 
	
\section{Acknowledgements}

	We thank Fran\c{c}ois Gould, Dean Adams, David Polly, Gary Bronner, Steve Brusatte, Steve Wang, Luke Harmon, Thomas Guillerme and the members of NERD club for insightful discussions and the museum staff and curators for their support and access to collections. Funding was provided by an Irish Research Council EMBARK Initiative Postgraduate Scholarship (SF) and the European Commission CORDIS Seventh Framework Programme (FP7) Marie Curie CIG grant. Proposal number: 321696 (NC)

\bibliographystyle{jeb}
\bibliography{refs_disparity}
% I downloaded the jeb.bst file from http://schneider.ncifcrf.gov/latex.html but there isn't an associated style file

% NC: You can make one via the command line as I showed you in one of my LaTeX lessons.

	%**********************
	%SF: I still need to do this so that the references have the abbreviated journal titles, don't include the doi and so that book references don't have the total number of pages but do have the editors' names
	%*******************************************




\end{document}