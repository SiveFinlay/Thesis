\chapter*{Abstract}
\chaptermark{abstract}
\addcontentsline{toc}{chapter}{Abstract}

	Patterns of morphological diversity (the variety of physical form) are interesting for what they tell us about species' evolutionary histories and ecological interactions. Some groups are more morphologically diverse than others and studying these clades provide an insight into ecological relationships and evolutionary pressures. For a more complete understanding of morphological diversity, it is important to take a quantitative approach instead of relying on subjective, qualitative assessments. 
	
	Tenrecs (Afrosoricida, Tenrecidae) are a family of small mammals which is often cited as an example of an exceptionally morphologically diverse group. However, this assumption has not been tested. In this thesis, I use geometric morphometric analyses of skull shape to test whether tenrecs are more morphologically diverse than their closest relatives, the golden moles (Afrosoricida, Chrysochloridae). Tenrecs occupy a wider range of ecological niches than golden moles so I predicted that they would be more morphologically diverse. 
	
	Contrary to expectations, I found that tenrec skulls are only more morphologically diverse than golden moles when measured in lateral view. Furthermore, the similarities among the species-rich \textit{Microgale} tenrec Genus appear to mask higher morphological diversity in the rest of the Family. 
	
	My results reveal new insights into the morphological diversity of tenrecs and highlight the importance of using quantitative methods to test qualitative assumptions about patterns of morphological diversity. In addition, the extensive morphological data set which I collected for tenrecs and other small mammal species represents a significant resource for future research into the Family.
	
