\chapter{Discussion}
\label{chap:discussion}

\noindent

Tenrecs are often cited as an example of a mammalian group with high morphological diversity \citep{Olson2013, Soarimalala2011, Eisenberg1969}. They are also more ecologically diverse than their closest relatives \citep{Soarimalala2011, Bronner1995} so I predicted that they would be more morphologically diverse than golden moles. However, my results (\textbf{Chapter \ref{chap:disparity}}) do not support my original prediction, highlighting the importance of quantitative tests of perceived morphological patterns.

\section{Morphological diversity in skulls}

	In my full analysis, tenrecs only had higher morphological diversity than golden moles when the skulls were measured in lateral view (Table \ref{tab:diversity}). There was no difference in morphological diversity when I analysed the skulls in dorsal or ventral views. This is most likely due to my choice of landmarks. The two outline curves in lateral view (Figure \ref{fig:sklat_mands}) emphasise morphological variation in the back and top of the skulls. These curves summarise overall shape variation but they do not identify clear anatomical differences because they are defined by relative features rather than homologous structures \citep{Zelditch2012}. Therefore, high morphological diversity in tenrecs when analysed in this view may not indicate biologically or ecologically relevant variation.	
	These lateral aspects of the skull morphology were not visible in the dorsal and ventral photographs so they could not be included in those analyses. In contrast, my landmarks in the dorsal, and particularly ventral, views focus on morphological variation in the overall outline shape of the sides of the skull and palate (Figure \ref{fig:skdors_skvent}). The result that tenrecs are no more diverse than golden moles in these areas makes intuitive sense: most tenrecs have broad, non-specialised diets \citep{Olson2013} so there is no obvious functional reason why they should have particularly diverse palate morphologies.
	The different results for my analysis of lateral skull morphologies compared to dorsal and ventral views highlight the importance of using multiple approaches when studying 3D morphological shape using 2D geometric morphometrics techniques \citep{Arnqvist1998}.

	
%Why does subsampling the Microgale make a difference
	In addition to the differences among the three skull views, my results indicate that, in skulls at least, the overall morphological diversity within tenrecs is not as large as is often assumed \citep[e.g.][]{Eisenberg1969, Olson2013}. Studies of morphological variation are sensitive to the sampling used. If a particular morphotype is over-represented then the similarities among those species will reduce the overall morphological variation within the group \citep{Foote1991}. This appears to be the case for my data; it was only when I included a sub-sample of \textit{Microgale} tenrecs that I found higher morphological diversity in tenrecs compared to golden moles across all three skull analyses (Table \ref{tab:diversity}). %NC: The sentence I deleted was pretty much the same as the next one in terms of meaning.
	While there are clear physical differences among Family members \citep{Olson2013, Eisenberg1969}, the majority of tenrecs are very morphologically similar \citep{Jenkins2003} so morphological diversity in the Family as a whole is not as large as it first appears. Of course my results are based on skull shape only and analyses of other morphological traits may produce different results (see section \ref{sect:caveats}), but they do provide an insight into the differences between subjective and quantitative assessments of morphological diversity.  

\section{Morphological diversity in mandibles}
	My analyses of morphological diversity in tenrec and golden mole mandibles yielded very different results to those for the skulls. In the full comparison (31 species of tenrec and 12 species of golden mole), I found that golden moles have more morphologically diverse mandible shapes than tenrecs (Table \ref{tab:diversity}). This difference was not significant with the reduced data set (Table \ref{tab:diversity}), supporting my finding that morphological similarities within the \textit{Microgale} Genus mask higher diversity within the rest of the Family.
	
	It is not clear why golden moles appear to have more morphologically diverse mandibles than tenrecs. Golden moles have generalised, insectivorous diets \citep{Bronner1995} which are no more specialised or variable than tenrecs \citep{Soarimalala2011}. Their fossorial mode of life does not offer an obvious explanation either as they dig using their forelimbs rather than their jaws.
	To gain further insight, I identified which aspects of mandible morphology contributed to the higher diversity in golden moles. My mandible landmarks and curves focus particular attention on the ascending ramus (condyloid, condylar and angular processes; Figure \ref{fig:sklat_mands}). When I repeated my analysis of morphological diversity without these three semilandmark curves, I found no significant difference between the two Families (section \ref{sect:results}). Therefore, my results seem to indicate that golden moles have greater morphological variation in the posterior structures of their mandibles compared to tenrecs. 
	
	One might expect that golden moles with highly disparate posterior mandible morphologies should also show high variability in the corresponding mandible articulation areas of the skull \citep[although developmental genetics studies have revealed that mandibles can also develop shape variation independently of skulls; ][] {Rot-Nikcevic2007}. However, I could not locate reliable, homologous points on the relevant areas of the skull in lateral view to test whether golden moles might be more variable than tenrecs in this region. Like my analyses of lateral skull morphologies, the semilandmark curves I used to summarise mandible shape variation are based on relative (type 3) landmarks rather than defined anatomical points \citep{Zelditch2012}, so discrepancies associated with this approach could offer another explanation for the unexpected results. Further investigation is required to determine whether apparently high morphological diversity in golden mole mandibles is biologically meaningful or a methodological artefact. 

		
\section{Caveats}
\label{sect:caveats}

	As highlighted above, landmark choice and placement will inevitably influence the results of a geometric morphometrics study. My interest in broad-scale, cross-taxonomic comparisons of cranial morphology constrained my choice of landmarks to those that could be accurately identified in many different species \citep[e.g.][]{Ruta2013, Goswami2011, Wroe2007}. In contrast, studies that use skulls to characterise morphological variation within species \citep[e.g.][]{Blagojevic2011, Bornholdt2008} or to delineate species boundaries within a clade \citep[e.g.][]{Panchetti2008} tend to focus on more detailed, biologically homologous landmarks \citep{Zelditch2012}. Repeating my analyses with a narrower taxonomic focus may give greater insight into the specific morphological differences among subgroups of tenrecs and golden moles.
	
	The goal of my study was to quantify morphological variation in tenrecs instead of relying on subjective assessments of their high morphological diversity \citep{Olson2013, Soarimalala2011, Eisenberg1969}. However, it is difficult to quantify overall morphological diversity because any study is inevitably constrained by its choice of specific traits \citep{Roy1997}. Variation in skull shape is only one aspect of overall morphology. Quantifying variation in other morphological traits could yield different patterns. Therefore future work should extend my approach beyond skulls to gain a more complete understanding of the overall morphological diversity of tenrecs and golden moles. Some of the additional data I collected (see Appendix \ref{appendix}) could be useful for this future work.



\section{Conclusions and Future directions}
\label{sect:concl}

	I have presented the first quantitative investigation of morphological diversity in tenrecs. My results indicate that, overall, tenrec skulls are not more morphologically diverse than golden moles and that similarities among the species rich \textit{Microgale} tenrecs mask signals of higher morphological diversity among the rest of the Family. These findings provide a foundation for future investigations of morphological diversity in the tenrec Family. For example, the additional data that I collected (see Appendix \ref{appendix}) could be used to assess the degree of convergent similarities among tenrecs and other small mammals. This would be the first quantitative test of whether apparent similarities \citep{Olson2013, Soarimalala2011, Eisenberg1969} should be treated as true convergence \citep[e.g.][]{Losos2011, Stayton2008}.
	Of course the results presented here are restricted to just one axis of morphological variation and further analysis of other traits is required. However, my findings represent a significant step towards a more quantitative understanding of patterns of morphological and evolutionary diversity in tenrecs. 

%I left out sections on measuring adaptiveness of phenotypic traits/adaptive radiation, ecological convergence and possible behavioural convergence (echolocation) because I didn't think it was relevant to introduce so much new material at the end of the thesis. But maybe it needs to be expanded more?

% NC: Seems fine to me. Good discussion of what you actually did. Anything else would be a bit waffly. 





