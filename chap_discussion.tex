\chapter{Discussion}
\label{chap:discussion}



\noindent
Overall summary of points and implications of findings

\section{Skulls}
%Why do you get different results for lateral compared to other skull views
%Why does subsampling the Microgale make a difference

\section{Mandibles}
%Why do golden moles seem to be more diverse than tenrecs: don't have a good answer, probably an issue with landmark selection (maybe this shouldn't be a separate section?)

\section{Potential issues}
%Difficulties and caveats: body size, trait choice, morphological proxy (I have this paragraph from my paper discussion).

\section{Future directions}
%These are just current ideas for future directions (based on PhD plans)
%I will definitely reduce them down

%Measuring the adaptiveness of phenotypic traits
%More discussion of adaptive evolution, adaptiveness of traits

%Evolutionary predictions of divergence and convergence
%Simulations - use some of the methods that I have and the advice from David Polly

%Ecological convergence


\subsection{\normalfont{Studies of behavioural convergence}}
%This section is from my original steering committee report
%I will definitely edit it down but I could also remove it completely, especially since my thesis is now about morphological diversity and not convergence in general
	Studies of morphological (phenotypic) similarities among tenrecs and other small mammals could be extended to tests of functional or behavioural convergences among tenrecs and other distantly related species.
	%References about why this is important: maybe the shark vs. dolphin swimming papers?
	One particularly interesting extension would be to study echolocatory capabilities in tenrecs.

	Gould \citeyear{Gould1965} demonstrated echolocatory abilities in three species of tenrec; \textit{Echinops telfairi, Hemicentetes semispinosus} and \textit{Microgale dobsoni}, indicating that they share behavioural similarities with some shrews \citep{Gould1964, Tomasi1979, Siemers2009}. Subsequent work demonstrated that the auditory range of \textit{Echinops telfairi} includes ultrasonic frequencies \citep{Drexl2003} and there have also been physiological investigations of stridulation behaviour in \textit{Hemicentetes} tenrecs \citep{Eisenberg1969, Endo2010}. However, aside from these studies echolocatory capabilities in tenrecs have not been investigated further. Recent studies have found evidence for sequence-level genomic convergence underlying independent origins of echolocation in multiple mammalian lineages \citep{Parker2013}. Therefore, re-assessing and expanding behavioural echolocatory capabilities within tenrecs could be an important first step towards looking for further convergence at the genetic level.

	I went to Madagascar in March/April 2014 as part of a research trip led by Dr.Steve Goodman to conduct behavioural tests of echolocation in \textit{Microgale}. My aim was to record the sounds made by the animals as they moved through a wooden maze towards a food reward to determine whether there was evidence that they were using sounds to navigate through their environment. I tried multiple variations of our protocol but unfortunately none of the animals we tested produced any noise (17 individuals from 5 different species). However, it is clear that this negative result is a failure of the experimental design rather than an indication that \textit{Microgale} don't navigate using sounds. My sample included \textit{Microgale dobsoni} which is one of the few species that is known to echolocate from previous experiments \citep{Gould1965}. Similarly, other more experienced researchers in the group had heard the \textit{Microgale} making sounds while foraging. 
	
	Previous studies of echolocation in small mammals \citep{Gould1964, Gould1965, Tomasi1979, Siemers2009} all used captive individuals which were trained to perform specific tasks. Unfortunately such a prolonged procedure was not possible for me within constraints of time and facilities. However similar, more prolonged studies could reveal very interesting insights into echolocatory behaviour and capabilities within tenrecs. This work would also fit in with a more holistic view of understanding convergence among tenrecs and other small mammals within morphological, ecological, behavioural and potentially genetic contexts.



\section{Conclusions}



