\chapter{Disparity in tenrecs compared to their closest relatives}
\label{chap:disparity}

\section{Introduction}

\section{Methods}
%I copied most of this section from the methods part of my disparity paper

	For each of the morphometric data sets (dorsal, ventral, lateral skulls and mandibles), I ran a general Procrustes alignment (section \ref{sect:procrustes}) of just my tenrec and golden mole specimens to compare morphological diversity in the two families. I used the principal components axes which accounted for 95 \% of the cumulative variation to calculate four disparity metrics; the sum and product of the range and variance of morphospace occupied by each family \citep{Brusatte2008, Foth2012, Ruta2013}. We also calculated morphological disparity directly from the Procrustes-superimposed shape data based on the sum of the squared inter-landmark distances among species pairs \citep[SSqDist,][]{Zelditch2012}.
	
	I used two approaches to test whether tenrecs have significantly different morphologies compared to golden moles. The first was a comparison of morphospace occupation between the two groups with non parametric MANOVAs \citep{Anderson2001} to test whether tenrecs and golden moles occupy significantly different areas of morphospace \citep[e.g][]{Serb2011, Ruta2013}. 
	
	Secondly, I used pairwise permutation tests to test the null hypothesis that tenrecs and golden moles have equal disparity. If this hypothesis were true then the designation of each species as belonging to either tenrecs or golden moles should be arbitrary because each group would have the same disparity. Therefore I permutated the data by assigning family identities at random to each specimen and calculated the differences in disparity for each of the new family groupings. I repeated these permutations 1000 times to generate a null distribution of the expected differences in family disparity. I compared the observed (true) measures of the differences in disparity between tenrecs and golden moles to these permutated distributions to test whether the families had significantly different levels of disparity.

	The majority of tenrec species (19 out of 31 in my data) are members of the \textit{Microgale} (shrew-like) genus which is notable for its relatively low phenotypic diversity \citep{ Soarimalala2011, Jenkins2003}. The strong similarities among these species may mask signals of higher disparity among other tenrecs. Therefore I repeated my family-level comparisons of disparity with a reduced data set that excluded the \textit{Microgale} so that I could compare disparity within the remaining 12 tenrec species to disparity within the 12 species of golden moles.

\section{Results} %Copied this in from the disparity paper

	When we compared tenrecs’ cranial morphologies to their closest relatives we found a trend towards higher disparity in tenrecs than
	in golden moles. However, these apparent differences were only
	significant for some disparity metrics. In contrast, golden moles have more diverse mandible shapes than tenrecs which appears to be due to greater morphological variety in the posterior mandible strucutres of golden moles (section \ref{sect:nonmic_gmoles}).

\subsection{\normalfont{Morphological disparity in tenrecs and golden moles}}
	Figures  \ref{fig:fourPCA} depict the morphospace plots derived from our principal components analyses of average Procrustes-superimposed shape coordinates for each species in our skull and mandible data respectively. We used the principal components axes which accounted for 95\% of the cumulative variation (n = 7, 8, 8 axes for the dorsal, ventral and lateral skull analyses respectively and n = 12 axes for the mandibles) to calculate the disparity of each family. 

%-----------------------------------------------------
%PCA figures: same one that's in the disparity paper
	%Source of this figure in my shape_data/output within the disparity folder 
	\begin{figure}[h]
	\centering
	\includegraphics[width=1\linewidth]{Disparity/writing/figures/FourPlotPCA.png}
	\caption[Principal components plots of the morphospaces occupied by tenrecs and golden moles]
		{Principal components plots of the morphospaces occupied by tenrecs (red, n=31 species) and golden moles (black, n=12) for the skulls: dorsal (A), ventral (B), lateral (C) and mandibles (D) analyses. Axes are PC1 and PC2 of the average scores from a PCA analysis of mean Procrustes shape coordinates for each species. }
	\label{fig:fourPCA}
	\end{figure}
%--------------------------------------------------------

	Tenrecs and golden moles clearly have very different cranial and mandible morphologies: in each analysis, the families occupy significantly different areas of morphospace (npMANOVA, table \ref{tab:npmanova.summary}). 

%------------------------------------------------
%Summary table of the npMANOVA morphospace occupation comparisons	
	\begin{table}[!htb]	%!htb keeps the table within this section		

	\caption[Comparisons of morphospace occupation by tenrecs and golden moles (npMANOVA)]
		{Summary of the npMANOVA comparisons of morphospace occupation for tenrecs and golden moles in each of the four analyses (three views of skulls and mandibles). In each case the two families occupy significantly different areas of morphospace.}
	\centering
	\input{Disparity/writing/tables/npmanova.summary} 
	\label{tab:npmanova.summary}  
	\end{table}
%---------------------------------------
	Our comparisons of disparity levels within each family yielded different trends for the skulls compared to the mandible analyses.	
	In our analyses of the three different views of the skulls, when disparity is calculated from principal component - based metrics there is there is an overall trend for tenrecs to have higher disparity than golden moles. However, none of these differences are statistically significant (table \ref{tab:disp.summary}). In contrast, when we calculated disparity based on the sum of squared interlandmark differences between species pairs \citep{Zelditch2012} then golden moles had significantly higher levels of disparity than tenrecs (table \ref{tab:disp.summary}).

\bigskip
%------------------------------------------
%Summary table of the family comparisons for all tenrecs vs. golden moles
	\begin{table}[!htb]			
	\caption[Comparisons of disparity metrics for tenrecs and golden moles]
		{Summary of disparity comparisons between tenrecs (T) and golden moles (G) for each of the data sets(rows) and five disparity metrics (columns). "Mandibles:one curve" refers to my shape analysis of mandibles excluding the three curves around the posterior structures of jaw (figure \ref{fig:sklat_landmarks}). Significant differences are highlighted in bold with the corresponding p value in brackets. Disparity metrics are; sum of variance, product of variance, sum of ranges, product of ranges and sum of squared distances among species. }
	\centering
	%Disparity family comparison results summary
%All tenrecs and golden moles

\begin{tabular}[t]{l l l l l l }		
\hline
\textbf{Disparity metric} & \textbf{SumVar} & \textbf{ProdVar} & \textbf{SumRange} & \textbf{ProdRange} & \textbf{SSqDist} \\
\hline
Skulls dorsal & T$>$G & T$>$G & T$>$G & T$>$G &	\textbf{G$>$T* (0)}\\
%-----------------------------------------------------------
Skulls lateral	& T$>$G & T$>$G & T$>$G & T$>$G & \textbf{G$>$T* (0)}\\
%-------------------------------------------------------
Skulls ventral & T$>$G & G$>$T & T$>$G & T$>$G & \textbf{G$>$T* (0)}\\
%-------------------------------------------------------
Mandibles & G$>$T & \textbf{G$>$T* (0.008)} & \textbf{T$>$G* (0.025)} & \textbf{T$>$G* (0.009)} &	\textbf{T$>$G* (0)}\\
%-------------------------------------------------------
Mandibles & G$>$T & G$>$T & T$>$G & T$>$G &	\textbf{T$>$G* (0)}\\
%-------------------------------------------------------
\hline
\end{tabular} 
	\label{tab:disp.summary}  
	\end{table}
%------------------------------------------
\bigskip
	There is a less clear pattern from our analysis of disparity in the mandibles. Three of our five metrics indicate that golden moles have significantly higher disparity in the shape of their mandibles than tenrecs (table \ref{tab:disp.summary}) although one metric (sum of ranges) indicated the opposite result. 
	
	The three curves that we placed at the back of the mandibles (figure \ref{fig:mands_landmarks}) place a particular emphasis on shape variation in the posterior of the bone; the ramus, coronoid, condylar and angular processes. Therefore, higher disparity in golden mole mandibles compared to tenrecs could be driven by greater morphological variation in these structures. To test this idea, we repeated our morphometric analyses of the mandibles with a reduced data set of points; just the seven landmark points and one single curve at the base of the jaw between landmarks 1 and 7 (figure \ref{fig:mands_landmarks}). When we compared familial disparity levels with this reduced data set we found that golden moles no longer had significantly higher disparity than tenrecs but rather there were some indications that the opposite was true (table \ref{tab:disp.summary}).
	
\subsection{\normalfont{Morphological disparity in non-\textit{Microgale} tenrecs and golden moles}}
\label{sect:nonmic_gmoles} 	   
	
	We repeated our disparity comparisons with a subset of the tenrec specimens to remove the large and phenotypically similar \textit{Microgale} tenrec genus. In this case we found that tenrecs have significantly higher disparity than golden moles when the skulls are analysed in lateral view (table \ref{tab:disp.nonmic.summary}). However, none of the other comparisons in any of the analyses were significant. Similarly, the trend in the main analysis for golden moles to have significantly higher disparity measured as the sum of squared inter-landmark distances (table \ref{tab:disp.summary}) was not repeated in this comparison of disparity in non-\textit{Microgale} tenrecs and golden moles (table \ref{tab:disp.nonmic.summary}).
\bigskip
%-------------------------------------------------------
%Summary of the family comparisons for non-Microgale tenrecs vs. golden moles
	\begin{table}[!htb]			
	\caption[Comparisons of disparity metrics for non-\texit{Microgale} tenrecs and golden moles]
		{Summary of disparity comparisons between non-\textit{Microgale} tenrecs (T) and golden moles (G) for each of the data sets(rows) and five disparity metrics (columns). Significant differences are highlighted in bold with the corresponding p value in brackets. Disparity metrics are; sum of variance, product of variance, sum of ranges, product of ranges and sum of squared distances among species. }
	\centering
	%Disparity family comparison results summary
%Non-Microgale tenrecs and golden moles

\begin{tabular}[t]{l l l l l l }		
\hline
\textbf{Disparity metric} & \textbf{SumVar} & \textbf{ProdVar} & \textbf{SumRange} & \textbf{ProdRange} & \textbf{SSqDist} \\
\hline
Skulls dorsal & T$>$G & T$>$G & T$>$G & T$>$G &	T$>$G\\
%-----------------------------------------------------------
Skulls lateral	& \textbf{T$>$G* (0.014)} & T$>$G & \textbf{T$>$G* (0.001)} & \textbf{T$>$G*(0.003)} & \textbf{G$>$T* (0.014)}\\
%-------------------------------------------------------
Skulls ventral & T$>$G & T$>$G & T$>$G & T$>$G & T$>$G\\
%-------------------------------------------------------
Mandibles & T$>$G & G$>$T & T$>$G & G$>$T &	G$>$T\\
%-------------------------------------------------------
%Don't need to do a subset analysis of mandibles for non-Microgale tenrecs
%Mandibles one curve & G$>$T & T$>$G & T$>$G & T$>$G & G$>$T\\
%-------------------------------------------------------
\hline
\end{tabular} 
	\label{tab:disp.nonmic.summary}  
	\end{table}
%------------------------------------	


\section{Discussion}





