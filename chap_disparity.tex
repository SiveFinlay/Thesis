\chapter{Morphological diversity in tenrecs compared to their closest relatives}
\label{chap:disparity}


\section{Introduction}
%SF: I haven't repeated the paper's introduction here because the information is in the introduction of my thesis instead: this chapter is a continuation of the main introduction and methods chapter


	It is important to study patterns of morphological diversity to gain a greater insight into species' evolutionary histories and ecological interactions (chapter \ref{chap:introduction}). In this chapter, I present the first quantitative investigation of morphological diversity in tenrecs. I use geometric morphometric techniques \citep{Rohlf1993} to compare cranial morphological diversity in tenrecs to their closest relatives, the golden moles. Tenrecs inhabit a wider variety of ecological niches than golden moles \citep{Soarimalala2011, Bronner1995} so I expect tenrecs to be more morphologically diverse than golden moles. 
	
	%I originally included the following paragraphs but then I thought that it sounded too much like an abstract
	
	%However, tenrecs only had higher morphological diversity than golden moles in some but not all views of skull shape. Further analyses (see section \ref{sect:results}) revealed that the morphological similarities within the \textit{Microgale} tenrec Genus appears to reduce the overall morphological diversity of the Family as a whole. In addition, I found that morphological diversity in mandible shape did not follow the same patterns as found in skulls.
	%These results highlight the importance of using quantitative methods to gain a greater understanding of patterns of morphological diversity.

%------------------------------------------------------
\section{Methods}

	The methods I used to measure morphological diversity involved several steps of data collection, processing and analysis. For clarity,  figure \ref{fig:flow} summarises all of these steps and I describe them in detail below. Note that I repeated the same analyses for each of the different sets of photographs: skulls in dorsal, ventral and lateral views and mandibles in lateral view.
	
%*************************************************
%Methods flowchart: 
%**************************************
	%SF: I've changed parts of this compared to the chart in the paper manuscript but the overall steps are the same because I'm hoping that the new description in the text makes them clearer.
		\begin{figure}[!htbp]
		\centering
		\includegraphics[width=1\linewidth,height=0.8\textheight]{Disparity/writing/figures/Methods_flowchart_thesis.png}
		
		\caption[Flowchart diagram of data collection and analysis]
			{Summary of the main steps in my data collection, processing and analysis protocol. Note that the analyses were repeated separately for each set of photographs: skulls in dorsal, ventral and lateral views and mandibles in lateral view. The dashed arrows refer to the stage at which I selected a subsample of the tenrecs (including just five species of the \textit{Microgale} Genus) so that I could compare the morphological diversity of this reduced subsample of tenrec species to the diversity of golden moles.}
		
		\label{fig:flow}
		\end{figure}
%----------------------------------------------------	

	I have already described how I photographed skulls and summarised their morphologies using landmark morphometrics (chapter \ref{chap:methods}, figure\ref{fig:flow}). In total, I photographed 99 different species from seven different mammal Families (table \ref{tab:species.measured}). However, here I am only using part of this full data set to compare morphological diversity of tenrecs to their closest relatives, the golden moles. Some skulls were partly damaged and therefore I could not compare them in all of the different views so the exact number of skulls that I used varies slightly for each analysis. I used photographs of 182 skulls in dorsal view (148 tenrecs and 34 golden moles), 173 skulls in ventral view (141 tenrecs and 32 golden moles), 171 skulls in lateral view (140 tenrecs and 31 golden moles) and 181 mandibles in lateral view (147 tenrecs and 34 golden moles). These samples represent 31 species of tenrec (out of the total 34 in the Family \citep{Olson2013}) and 12 species of golden moles (out of a total of 21 species in the Family, \citep{Asher2010}).
	
	%SF: My methods chapter describes all of the stages up to the PC analysis but I'm repeating some of that here so that I can refer to the diagram.

	After placing the landmarks and creating sliders files within the TPS software series (section \ref{sect:morphometrics}, figure \ref{fig:flow}), I combined the files into a single morphometrics data object and carried out all further analyses in R \citep{Team2014}. At this stage, I either used the full data set (31 species of tenrec and 12 species of golden mole) or a reduced data set with just 17 species of tenrec. 
	
	%This paragraph was at the end of the methods section in the paper but I think it makes more sense to move it up because otherwise the dashed lines in the flowchart don't become clear until the end of the section
	
	I created this reduced data set because the majority of tenrec species (19 out of 31 in my data) belong to the \textit{Microgale} (shrew-like) Genus that has relatively low morphological diversity \citep{Soarimalala2011, Jenkins2003}. This may mask signals of higher morphological diversity among other tenrecs. To test this, I created a subset of the tenrec data that included just five of the \textit{Microgale} species, each representing one of the five sub-divisions of \textit{Microgale} outlined by Soarimalala and Goodman \citeyearpar{Soarimalala2011}, i.e. small, small-medium, medium, large and long-tailed species. I compared the morphological diversity of this subset of tenrecs (n=17: five \textit{Microgale} and 12 non \textit{Microgale} species) to that of the 12 species of golden moles (dashed arrows in figure \ref{fig:flow}). After this selection stage, all further steps in the analyses were the same.
	
	For each analysis, I did a general Procrustes alignment of the shape data and calculated the mean shape coordinates for every species (see section \ref{sect:procrustes}). I used these average, Procrustes-superimposed shape coordinates for a principal components analysis (PCA) and then selected the number of principal component (PC) axes that accounted for 95\% of the variation in the data (figure \ref{fig:flow}).
	
\subsection{\normalfont{Calculating morphological diversity}}
	
	I grouped the PC scores for tenrecs and golden moles separately so that I could calculate the diversity of each Family and then compare the two groups (figure \ref{fig:flow}). I compared morphological diversity in two ways.
	
	First, I used non parametric multivariate analysis of variance (npMANOVA) \citep{Anderson2001} to test whether tenrecs and golden moles occupied significantly different positions within the morphospaces defined by the PC axes that accounted for 95\% of the overall variation in the data \citep[e.g][]{Serb2011, Ruta2013}. A significant difference between the two Families would indicate that they have unique morphologies which do not overlap.
	
	Secondly, I compared morphological diversity within tenrecs to the diversity within golden moles. I define diversity as the mean Euclidean distance (sum of squared differences) between each species and its Family centroid (figure \ref{fig:centroids}). 
	If tenrecs are more morphologically diverse than golden moles, then they should be more dispersed within the morphospaces. 
	
%-----------------------------
%Diagram of mean distance to centroid (based on the diagram in Cooper et al 2009)
	\begin{figure}[!htbp]
	\centering
	\includegraphics [width=0.7\linewidth, height=0.7\textheight, keepaspectratio]{Disparity/writing/figures/Centroids.png}
	\caption[Calculating diversity as mean Euclidean distance to Family centroid.]
		{Calculating morphological diversity as the mean Euclidean distance between each species and the Family centroid. Every species has scores on the principal components (PC) axes that account for 95\% of the variation in the Principal Components Analysis. The number of axes (PCn) varied for each analysis but they are the same within a single analysis. PC scores are used to calculate the Euclidean distance from each species to the Family centroid (average PC scores for the entire Family). Morphological diversity of the Family is the average value of these Euclidean distances.}
	\label{fig:centroids}
	\end{figure}
	
%--------------------------------------

	
	The Families have unequal sample sizes: 31 (or a subset of 17) tenrec species compared to just 12 golden mole species. Morphological diversity is usually decoupled from taxonomic diversity \citep[e.g.][]{Ruta2013, Hopkins2013} so larger groups are not necessarily more morphologically diverse. However, comparing morphological diversity in tenrecs to the diversity of a smaller Family could still bias the results. I used pairwise permutation tests to account for this potential issue. 

%SF: I've re-phrased this description (based on Steve Wang's explanation in his email which I can send to you if you like) so hopefully it's clearer

	I tested the null hypothesis that tenrecs and golden moles have the same morphological diversity (the same mean Euclidean distance to the Family centroid). If this is true then the label identifying each specimen as a tenrec or golden mole is arbitrary. In this case, you could randomly assign the group identity of each species (i.e. shuffle the "tenrec" and "golden mole" labels) and then re-compare the morphological diversity of the two groups: there would still be no difference. I repeated this shuffling procedure (random assignation of group identity) 1000 times and calculated the difference in morphological diversity between the two groups for each permutation. This generated a distribution of 1000 values which are calculations of the differences in morphological diversity under the assumption that the null hypothesis (equal diversity in the two Families) is true. This method automatically accounts for differences in sample size. Shuffling of the group labels preserves the sample size of each group: there will always be 12 species labelled as "golden mole" and then, depending on the analysis, either 31 or 17 species labelled as "tenrec". Therefore, the 1000 permuted values of differences in morphological diversity create a distribution of the expected difference in diversity between a group of sample size 31 (or 17 in the case of the subsetted tenrec data) compared to a group of sample size 12 under the null hypothesis that the two groups have the same morphological diversity. I compared the observed measures of the differences in morphological diversity between the two Families to these null distributions to determine whether there were significant differences after taking sample size into account (two tailed test).


	
%----------------------------------------------------
\newpage
\section{Results}
\label{sect:results}
	%Edited these from the paper comments
	%Change to using all four PCA plots

	Figure \ref{fig:FourPCA} depicts the morphospaces defined by the first two principal component (PC) axes from my principal components analyses (PCAs) of skull and mandible morphologies. The PCAs are based on the average Procrustes-superimposed shape coordinates for skulls in three views (dorsal, ventral and lateral) and mandibles in lateral view.

%-----------------------------
%New PCA figure: all four PCAs together, pictures as legend, bigger axis labels
%I haven't put in the warp diagrams because they made it too crowded. I could add warps if I put in each PCA as a separate plot
	
%PCA figure: Just sklat as an example
	%From the twofamily_disparity_PCAplots script
	\begin{figure}[!htbp]
	\centering
	\includegraphics[width=1\linewidth, height=1\textheight, keepaspectratio]{Disparity/writing/figures/FourPCA_shapes.png}
	\caption[Morphospace (principal components) plot of morphological diversity in lateral views of tenrec and golden mole skulls.]
		{Principal components plots of the morphospaces occupied by tenrecs (triangles, n=31 species) and golden moles (circles, n=12 species) for the skulls (dorsal, ventral and lateral views) and mandibles (lateral view). Each point represents the average skull shape of an individual species. Axes are principal component 1 and principal component 2 of the average scores from Principal Components Analyses of mean Procrustes shape coordinates for each species.}
	\label{fig:FourPCA}
	\end{figure}
%----------------------------------------------

	To compare morphological diversity in the two families, I used the PC axes which accounted for 95\% of the cumulative variation in each of the skull analyses: dorsal (n=6 axes), ventral (n=7 axes) and lateral (n=7 axes) and the mandibles (n=11 axes).
	
	First, I compared the position of each Family within the morphospace plots. Tenrecs and golden moles occupy significantly different positions in the dorsal (npMANOVA, F \textsubscript{1,42} = 68.13, R$^2$ =0.62, p=0.001 ), ventral (npMANOVA, F \textsubscript{1,42} = 103.33, R$^2$ =0.72 , p=0.001 ) and lateral (npMANOVA, F \textsubscript{1,42} = 76.7, R$^2$ =0.652, p=0.001 ) skull morphospaces as well as in the mandible morphospace (npMANOVA, F \textsubscript{1,42} = 60.38, R$^2$ =0.59, p=0.001),  indicating that the Families have very different, non-overlapping cranial and mandible morphologies (figure \ref{fig:FourPCA}). 
	
	%Numbers are from the npMANOVA based on PC axes within my diversity_twofamily_cent_dist script
%----------------------------------------
%Results table: I changed the format to make it fit better
	\begin{table}[!htbp]			
		\caption[Comparison of morphological diversity in tenrecs and golden moles.]
		{Morphological diversity in tenrecs compared to golden moles. I repeated each analysis with the full data (31 tenrec species) and then with 17 tenrec species (including just five species from the \textit{Microgale} genus). In each case, I compared the morphological diversity in tenrecs to the diversity within 12 species of golden moles. Significant differences between the two Families (p $<$ 0.05) are highlighted in bold.}
		%Diversity based on centroid distances results summary
%All tenrecs and golden moles
%Morphological diversity based on comparing the mean Euclidean distances to each family's centroid
%NB: Journal of Mammalogy asks for the degrees of freedom that accompany the t test but I'm getting different degrees of freedom in each summary and I don't know why

%Re-ordered the table so that everything would fit in better


\resizebox{\columnwidth}{!}{
%Scales down the table to fit within the column width
	% If this is too small then I'll probably need to break the table into two
\begin{tabular}{c l c c c c}		
\hline
Tenrec species & Analysis & Tenrecs  & Golden moles & t & p \\
%--------------------------------
& & (mean$\pm$ s.e) & (mean$\pm$ s.e) & &\\
\hline
%\multicolumn{1}{l}
%---------------------------
 31 & Skulls dorsal & \multicolumn{1}{l}{0.036 $\pm$ 0.0029} & 0.029 $\pm$ 0.0032 & -1.63 & 0.11 \\
%--------------------------------------
 & Skulls ventral & \multicolumn{1}{l}{0.048 $\pm$0.0034} & 0.044 $\pm$ 0.0041 & -0.676 & 0.51\\
%-----------------------------------------
 & Skulls lateral & 0.044 $\pm$ 0.0041 & 0.032 $\pm$ 0.0037 & -2.16 & \textbf{0.04}\\
%----------------------------------------
 & Mandibles & 0.049 $\pm$ 0.0044 & 0.067 $\pm$ 0.0054 & 2.62 & \textbf{0.014}\\
%--------------------------
\hline
%-----------------------------------------
17 & Skulls dorsal & 0.044 $\pm$ 0.0025 & \multicolumn{1}{l}{0.029 $\pm$0.003} & -3.62 & \textbf{0.001}\\
%---------------------------------
 & Skulls ventral & \multicolumn{1}{l}{0.054 $\pm$ 0.004} & \multicolumn{1}{l}{0.042 $\pm$ 0.004} & -2.23 & \textbf{0.04}\\
%-------------------------------------
 & Skulls lateral &  \multicolumn{1}{l}{0.054 $\pm$ 0.005} & 0.031 $\pm$ 0.0037 & -3.47 & \textbf{0.002} \\
%--------------------
 & Mandibles & 0.055 $\pm$ 0.0049 & \multicolumn{1}{l}{0.062 $\pm$ 0.005} & 1.003 & 0.325 \\
%--------------------

\hline
\end{tabular}
} 
		\label{tab:diversity}  
	\end{table}
%------------------------------------

	Secondly, I compared the morphological diversity within each Family. Based on my measures of mean Euclidean distance to the Family's centroid, tenrec skulls are more morphologically diverse than golden mole skulls when they are measured in lateral view but not in dorsal or ventral view (table \ref{tab:diversity}). In contrast, when I analysed morphological diversity of skulls within the sub-sample of 17 tenrecs (including just five \textit{Microgale} species) compared to the 12 golden mole species, I found that tenrec skulls were significantly more morphologically diverse than golden moles in all analyses (table \ref{tab:diversity}).
		
	The results of my analyses of the mandibles were different to those for the skulls. In the full analysis (31 species of tenrec compared to 12 species of golden mole), I found that golden moles have significantly more diverse mandible shapes than tenrecs but this difference was not significant when I used just 17 tenrec species (table \ref{tab:diversity})
	

	The pairwise permutation tests for each analysis confirmed that differences in morphological diversity were not artefacts of differences in sample size (table \ref{tab:permutations}).
	
%------------------------------------------------------
%Added the permutation results
\begin{table}[!htbp]			
	\caption[Results of the permutation tests]{Results of the permutation analyses which compared the observed differences in morphological diversity to a null distribution of expected results. I repeated the permutation comparisons for both the full (31 species of tenrec compared to 12 species of golden mole) and reduced (17 species of tenrec compared to 12 golden moles) data sets. In each case, the observed differences in morphological diversity were significantly different to the expected differences under a null hypothesis (significant p values). Therefore, the differences in morphological diversity between the two Families were not just artefacts of differences in sample size.}
	%Combined table of permutation results

\resizebox{\columnwidth}{!}{
\begin{tabular}[t]{l l c c c c c c c}		
\hline
%------------------------------------------
Tenrecs& Analysis & \multicolumn{5}{c}{Morphological diversity} & p value\\
%\hline
%-------------------------------------------
 &  & \multicolumn{3}{c}{Measured values}& \multicolumn{2}{c}{Permuted values} &  \\
\hline
& & Tenrecs & Golden moles & Difference & min. & max. & \\
\hline
%--------------------------------------------------------
31 & Dorsal & 0.036 & 0.029 & 0.007 & -0.011 & 0.0098 &  0.013 \\
%--------------------------------------------------------
& Ventral &  0.048 & 0.044 & 0.0036 & -0.014 & 0.013 &  0.023\\
%--------------------------------------------------------
& Lateral & 0.044 & 0.032 & 0.012 & -0.012 & 0.011 & <0.001 \\
%--------------------------------------------------------
& Mandibles & 0.049 & 0.067 & 0.018 & -0.008 & 0.009  & <0.001 \\
%-----------------------------
\hline
%--------------------------------------------------------
17 & Dorsal & 0.044 & 0.029 & 0.015 & -0.011 & 0.014 &  <0.001 \\
%--------------------------------------------------------
& Ventral &  0.054 & 0.042 & 0.013 & -0.017 & 0.019 &  0.023 \\
%--------------------------------------------------------
& Lateral & 0.054 & 0.0313 & 0.022 & -0.018 & 0.0186 & <0.001 \\
%--------------------------------------------------------
& Mandibles & 0.055 & 0.0623 & 0.007 & -0.012 & 0.011 & 0.038\\
%-----------------------------
\hline
\end{tabular}
} 
	\label{tab:permutations}  
\end{table}
	
%----------------------------------------------------

\section{Discussion}

%I wasn't sure how much discussion to put in here and how much to leave for the separate chapter.

%For the moment, I've just summarised the main results here and I'm keeping more detailed discussion (expansion of the paper discussion along with some additional points about mandibles) for the separate chapter.
%But should I have more detail here?

	Tenrecs are often cited as an example of a mammalian group with high morphological diversity \citep{Olson2013, Soarimalala2011, Eisenberg1969} and I expected them to be more morphologically diverse than their closest relatives. However, tenrecs were only more morphologically diverse than golden moles in one of the three skull analyses (lateral view; table \ref{tab:diversity}). The morphologically similar \textit{Microgale} tenrecs seem to mask high morphological diversity in the rest of the tenrec Family; reducing the data to include a sub-sample of this \textit{Microgale} species revealed that the remaining tenrecs were significantly more morphologically diverse than golden moles across all three skull analyses (table \ref{tab:diversity}). 
	
	In contrast to these results for the skulls, golden moles appear to have more diverse mandible morphologies than tenrecs (table \ref{tab:diversity}). I recognised that my landmarks and curves for the mandibles focus particular attention on the ascending ramus (condyloid, condylar and angular processes, figure \ref{fig:sklat_mands}). Therefore, I  deleted the three semilandmark curves around these structures and repeated my comparison of morphological diversity in the two Families. In this case, I found no significant differences in morphological diversity between the two Families (t-test, t=1.709, p=0.099). Therefore, my results seem to indicate that golden moles have greater morphological variation in the posterior structures of their mandibles compared to tenrecs.
	
	My results highlight the importance of using quantitative methods to test qualitative assumptions about patterns of morphological diversity. I discuss the results in more detail in chapter \ref{chap:discussion}.




