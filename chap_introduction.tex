\chapter{Introduction}
\label{chap:introduction}

%Current paragraphs are copied from the disparity manuscript

\noindent


\section{Patterns of morphological diversity}
	Patterns of diversity, \\
	morphological differences, \\
	geometric morphometrics overview		
		%Phenotypically diverse groups have long attracted the attentions of evolutionary biologists (REFS). Studies which quantify phenotypic variety have important implications for understanding the factors that contribute to high morphological diversity in some groups and not others (REFS). For example...


\section{Disparity}
%Exceptional diversity
%Adaptive radiations and morphological disparity

	There are many famous examples of adaptively radiated groups \citep{Gavrilets2009}. However, there has also been considerable debate about how adaptive radiations should be defined \citep{Glor2010, Losos2010a} based on the relative importance of speciation rate, species richness and morphological diversity. One particular issue is whether it is even meaningful to classify a particular group of species as an adaptive radiation or not since any classification relies on arbitrary distinctions between what is most likely a continua of characteristics which describe the diversity of a particular clade \citep{Olson2009}.
	
	However, despite the controversies and disagreements, there does seem to be a consensus that high morphological diversity is an important criteria for identifying a group of species as belonging to the adaptive radiation scale \citep{Losos2010a, Olson2009}. One way to test whether a group shows high morphological diversity is through sister taxa comparisons. For example, Losos and Miles (\citeyear{Losos2002}) used this approach to demonstrate exceptional diversity in some but not all clades of iguanid lizards.

\section{Convergence}
Long history of study\\
Repeatability of evolution\\
Methods of measuring convergence
%Or maybe put this section into the discussion instead?

\section{Tenrecs}
%Phylogenetic history, 
%phenotypic and ecological variety
	
	The tenrec family is comprised of 34 species, 31 of which are endemic to Madagascar \citep{Olson2013}. From a single common ancestor \citep{Asher2006}, Malagasy tenrecs diversified into a wide variety of descendant species which convergently resemble distantly related insectivore mammals such as shrews (\textit{Microgale} tenrecs), moles (\textit{Oryzorictes} tenrecs) and hedgehogs (\textit{Echinops} and \textit{Setifer} tenrecs) \citep{Eisenberg1969}. These convergent resemblances are so great that tenrecs used to be considered part of the general "insectivore" clade and only molecular studies revealed their true phylogenetic affinites within the Afrotherian mammals \citep{Stanhope1998}.  
	
	%More background information on tenrec studies here

	Tenrecs are often cited as an example of an adaptively radiated family which exhibits exceptional morphological diversity \citep{Soarimalala2011, Olson2003, Eisenberg1969}. However, this apparent exceptional diversity is based on subjective comparisons to other groups and it has not been tested. Here we present the first quantitative test of patterns of phenotypic diversity in tenrecs and examine how morphological diversity in tenrecs compares to their closest relatives, the golden moles (Afrosoricida, Chryscholoridae). 
	

	
	Tenrecs are also considered to be a clade which is highly convergent with other small mammal species ...
	%More of a convergence introduction here

\section{Structure \& contents of this thesis}
	
	In this thesis I quantify patterns of phenotypic diversity within tenrecs (chapter \ref {chap:disparity}) and morphological convergence among tenrecs and other small mammals (chapter \ref {chap:convergence}). In chapter \ref {chap:methods} I describe the data collection and general analyses which form the basis for my studies of disparity and convergence.  These analyses are the first systematic and quantitative studies of patterns of phenotypic diversity in the tenrec family and I discuss their implications in chapter \ref{chap:discussion}.	
 	My results reveal new insights into our understanding of morphological variety in tenrecs and prompt many new questions and possible avenues for further research.


