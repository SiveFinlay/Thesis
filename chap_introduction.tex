\chapter{Introduction}
\label{chap:introduction}


\noindent


\section{Patterns of morphological diversity}
%Morphology is intersting and has a long history of study

%Morphology tells us about patterns of evolution, functional diversity, niche partitioning

%Many qualitative studies but quantitative is important

%Difficult to quantify morphology: limited by choice of axis
	Morphological diversity is difficult to quantify. Studies are inevitably constrained to measure the diversity of specific traits rather than overall morphologies \citep{Roy1997}. Different trait axes (such as cranial compared to limb morphologies) may yield different patterns of morphological diversity \citep{Foth2012}.
%Geometric morphometrics help to overcome some limitations

%Quantifying morphology is interesting because of what it tells us about evolutionary history
	


\section{Morphological diversity and adaptive radiations}
%Examples of morphological diversity in adaptive radiations

	Studies of morphological diversity have important implications for our understanding of adaptive radiations: 'evolutionary divergence of members of a single phylogenetic lineage into a variety of different adaptive forms' \citep[Futuyama 1998, cired by][]{Losos2010}.
	There are many famous examples of adaptive radiations including Darwin's finches, Caribbean \textit{Anolis} lizards and cichlid fish \citep{Gavrilets2009}. Each of these cases are characterised by striking morphological diversity.
	
%Morphological diversity isn't everything	
	However, adaptive radiations are not defined by morphological diversity alone. Some authors argue that speciation rates and patterns of taxonomic diversity are equally if not more important than morphological variety for identifying adaptively radiated groups \citep{Glor2010, Losos2010a}. Furthermore it is difficult to identify a clade as an adaptive radiation or not because any classification will necessarily be based on arbitrary statistical cut-offs of particular measures of diversity, be they taxonomic, morphological or functional \citep{Olson2009}. Identifying and defining specific characteristics of adaptive radiations remains a challenge.
	
%Morphology is important but it's difficult to measure adaptiveness	
		
	Despite the controversies, there is a consensus that high morphological diversity is a unifying characteristic of adaptive radiations \citep{Losos2010a, Olson2009}. In particular, identifying high levels of morphological diversity in 'adaptive' (i.e. functionally significant) traits is an important aspect of identifying adaptively radiated clades \citep{Losos2010a}. For example, the number and relative length of toe lamellae in \textit{Anolis} lizards have functional significance which contributed to the adaptive radiation of this clade (REF).
	%Find better details for the Anolis reference 
	
%Quantifying morphological diversity is important as a first step 
	Even if the adaptive significance of traits is known, we need a measure of the degree of morphological diversity to be able to identify exceptionally diverse groups. One approach is through sister taxa comparisons. This has the advantage of comparing the morphological diversity of clades that have been evolving for the same amount of time since diverging from a common ancestor \citep{Losos2002}. Of course, this is a relatively limited measure of whether a group shows exceptional morphological diversity but it is a start.
	Therefore, studies of morphological diversity can be an important first step towards characterising an adaptive radiation.

\section{Morphological convergent evolution}
Long history of study\\
Repeatability of evolution\\
Methods of measuring convergence
%Or maybe put this section into the discussion instead?

\section{Tenrecs}
%Tenrecs are a diverse group
	Tenrecs (Afrosoricida, Tenrecidae) are an example of a morphologically diverse group \citep{Soarimalala2011, Eisenberg1969}. The Family is comprised of 34 species, 31 of which are endemic to Madagascar \citep{Olson2013}. Body sizes of tenrecs span three orders of magnitude (2.5 to > 2,000g) which is a greater range than all other Families, and most Orders, of living mammals \citep{Olson2003}.
	Within this vast size range there are tenrecs which convergently resemble shrews (Microgale tenrecs), moles (Oryzorictes tenrecs) and hedgehogs (Echinops and Setifer tenrecs) \citep{Eisenberg1969}. Their similarities include morphological as well as behavioural and ecological similarities (REF). 

%Interesting phylogenetic history
	These similarities are even more remarkable when you consider tenrecs' phylogenetic history. Tenrecs used to be classified within the general "insectivore" clade and only molecular studies revealed their true phylogenetic affinites within the Afrotherian mammals \citep{Stanhope1998}. Therefore, despite initial appearances, tenrecs are more closely related to elephants, manatees and aardvarks than they are to shrews, moles or hedgehogs. 

%Common example of both adaptive radiations and convergence

%Need a quantitative approach to test these ideas



	
 
	


	Tenrecs are often cited as an example of an adaptively radiated family which exhibits exceptional morphological diversity \citep{Soarimalala2011, Olson2003, Eisenberg1969}. However, this apparent exceptional diversity is based on subjective comparisons to other groups and it has not been tested. Here we present the first quantitative test of patterns of phenotypic diversity in tenrecs and examine how morphological diversity in tenrecs compares to their closest relatives, the golden moles (Afrosoricida, Chryscholoridae). 
	

	
	Tenrecs are also considered to be a clade which is highly convergent with other small mammal species ...


\section{Structure \& contents of this thesis}
	
	In this thesis I present the first quantitative study of patterns of morphological diversity in tenrecs. I compiled an extensive data set of morphological characteristics in tenrecs, their closest relatives (golden moles) and the mammals that they convergently resemble. Chapter \ref{chap:methods} includes details about my data collection and the general morphometrics analyses which form the basis for my study. I collected morphological data on the skulls, limbs and skins from 366 specimens representing 99 species of small mammals. 
	I used a subset of this data in chapter \ref{chap:disparity} to quantify the morphological diversity of tenrec skulls compared to their closest relatives, the golden mole. Finally, in chapter \ref{chap:discussion} I discuss the implications of my findings within the context of our understanding of tenrec evolution.	
 	My results reveal new insights into our understanding of morphological variety in tenrecs and prompt many new questions and possible avenues for further research.


