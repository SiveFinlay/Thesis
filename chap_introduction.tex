\chapter{Introduction}
\label{chap:introduction}

% NC: Feels like this needs a bit of an intro before going into separate sections...

% NC: It's good that you're using the one paragraph one idea thing, but it still needs to flow together. This feels quite disjointed. Also do you need all the subsections? I think the more sensible approach might be to open quite broadly and then mostly present a more fleshed out version of your disparity paper's introduction. Remember you are writing this as an MSc thesis so all one story, the disparity "paper" shouldn't be a stand alone piece within this like it would be in a PhD thesis.

% NC: Another slight issue is that each paragraph is very short (this makes it feel disjointed too) and there are a couple of paragraphs that all feel like they could be the opening paragraph. 

% NC: But on the plus side, all the info is here, it just needs rearranging!

\noindent

\section{Patterns of morphological diversity}

%Morphology is interesting
	Patterns of morphological diversity are one of the most interesting aspects of evolutionary biology. %NC: why? Bit of an odd statement to start with. You could just start with the second which is less subjective.
	Understanding why and how some groups of species are more morphologically diverse than others remains a central challenge. 
	% NC: central challenge to whom?
	
%Morphology tells us about patterns of evolution, functional diversity, niche partitioning
	Morphological diversity has important implications for a variety of ecological traits. For example, morphological characteristics of limbs inform us about the evolution of locomotory style \citep[e.g.][]{Bou1987} and the trophic niches associated with particular dental morphologies affect speciation and diversification rates through time \citep{Price2012}.
	%NC: I guess this is the "why" for the comment above 
	
%Qualitative vs. quantitative (from disparity paper)
	Apart from a few exceptions \citep[e.g.][]{Brusatte2008, Goswami2011, Ruta2013}, it is still common to study morphological diversity from a qualitative rather than quantitative perspective. However, it is important to quantify morphological diversity because it has implications for studies of adaptive radiations \citep{Losos2010}, convergent evolution \citep[e.g.][]{Muschick2012, Harmon2005} and our understanding of biodiversity \citep{Roy1997}.

%Difficult to quantify morphology: limited by choice of axis (paragraph copied from disparity paper)
	Yet morphological diversity is difficult to quantify. Studies are inevitably constrained to measure the diversity of specific traits rather than overall morphologies \citep{Roy1997}. Different trait axes (such as cranial compared to limb morphologies) may yield different patterns of morphological diversity \citep{Foth2012}. Furthermore, linear measurements of morphological traits can restrict our understanding of overall morphological variation \citep{Rohlf1993}.
	
%Geometric morphometrics
	Geometric morphometric approaches help to overcome some of the limitations of traditional morphological studies \citep{Adams2004}. Morphometric studies based on caliper measurements of particular features can only describe a limited set of distances, ratios and angles which often fail to capture the overall shape of a particular structure \citep{Slice2007}. Geometric morphometrics circumvents these issues by using a system of Cartesian landmark coordinates to defined anatomical points. This method captures more of the true, overall anatomical shape of particular structures \citep{Mitteroecker2009}.
	
%Tie together
	Quantifying morphological diversity through geometric morphometric approaches reveals insights into evolutionary patterns and processes through time. These methods are particularly applicable for investigations of two major evolutionary patterns: adaptive radiations and convergent evolution.
	%Needs a better link at the end
%----------------------------------------------------------
\section{Morphological diversity and adaptive radiations} 
%NC: is this section (and the next) needed given the questions you ended up asking? What would be easier I think is drop the subsections, and just use a condensed version of this to sell the "why is morphological diversity interesting" question. Then end with "we need to be able to quantify diversity first" which leads into your disparity stuff. We can discuss the best way to do this. I know this is partly borne out of your desire to demonstrate the stuff you've learned/read but trust me, everyone knows you know more than you get to write up. :). It still needs to be relevant to the question at hand.

%Examples of morphological diversity in adaptive radiations

	Studies of morphological diversity have important implications for our understanding of adaptive radiations: 'evolutionary divergence of members of a single phylogenetic lineage into a variety of different adaptive forms' \citep[Futuyama 1998, cited by][]{Losos2010}.
	There are many famous examples of adaptive radiations including Darwin's finches, Caribbean \textit{Anolis} lizards and cichlid fish \citep{Gavrilets2009}. Each of these cases are characterised by striking morphological diversity.
	
%Morphological diversity isn't everything	
	Of course adaptive radiations are not defined by morphological diversity alone. Some authors argue that speciation rates and patterns of taxonomic diversity are equally if not more important than morphological variety for identifying adaptively radiated groups \citep{Glor2010, Losos2010a}. Furthermore, it is difficult to identify a clade as an adaptive radiation or not because any classification will necessarily be based on arbitrary statistical cut-offs of particular measures of diversity, be they taxonomic, morphological or functional \citep{Olson2009}. Identifying and defining specific characteristics of adaptive radiations remains a challenge.
	
%Morphology is important but it's difficult to measure adaptiveness	
		
	However, despite the controversies, there is a consensus that high morphological diversity is a unifying characteristic of adaptive radiations \citep{Losos2010a, Olson2009}. In particular, identifying high levels of morphological diversity in 'adaptive' (i.e. functionally significant) traits is an important aspect of identifying adaptively radiated clades \citep{Losos2010a}. For example, the morphological traits which define ecomorph classes in the adaptive radiation of Caribbean \textit{Anole} % NC: Either Anolis in italics, or anole not in italics. Anole is the common name.
	lizards are closely linked to habitat use \citep{Losos1998}.

	
%Quantifying morphological diversity is important as a first step 
	Even if the adaptive significance of traits is known, we need to quantify morphological diversity to be able to identify exceptionally diverse groups. One approach is through sister taxa comparisons. This has the advantage of comparing the morphological diversity of clades that have been evolving for the same amount of time since diverging from a common ancestor \citep{Losos2002}. Of course, this is a relatively limited measure of whether a group shows exceptional morphological diversity but it is a start.
	Studies of morphological diversity can be an important first step towards characterising an adaptive radiation.
%-------------------------------------------------
\section{Morphological convergent evolution}

%Long history and why it's interesting
	Studying morphological diversity is also important for our understanding of convergent evolution: the "evolution of similar features independently throughout the tree of life" \citep[Futuyma 1998, cited in][]{Losos2011}. There are many famous examples of morphologically convergent groups including freshwater cichlid fish \citep{Muschick2012}, Caribbean \textit{Anole} lizards % NC: See comment above
	 \citep{Mahler2013} and convergence between placental and marsupial mammals \citep{Wroe2007}. Characterising convergence within these groups is interesting because it gives an insight into the relative repeatability of evolution \citep{Losos2011}.

%Why it's important to measure convergence

	Theoretical studies have demonstrated that the evolution of convergent phenotypes is not as surprising as it might appear \citep{Stayton2008}. Given this information, there has been increasing interest in developing quantitative approaches towards measuring convergence. These methods allow us to measure the degree of convergence among particular groups so that we can assess whether observed similarities are greater than what we would expect to have evolved by chance \citep[e.g.][]{Muschick2012}.
	
%Many methods of measuring convergence: important to extend the groups
	There are many different methods for quantifying morphological convergence. These include methods for measuring convergence across a phylogenetic tree \citep{Stayton2008}, between \textit{apriori}-identified species pairs \citep{Arbuckle2014, Muschick2012, Stayton2006} and combining morphological data with measures of ecological similarity to determine community or faunal convergence \citep{Ingram2013, Mahler2013, Moen2013, Melville2006}. Each method has strengths and weaknesses depending on the type of data used and the exact questions asked. However, they all require researchers to collect detailed morphological trait data before the methods can be applied. Unfortunately, this data is missing for many groups so the methods have been developed using very few convergent clades (predominantly cichlids and \textit{Anoles} %NC: See comment above
	). Therefore it is important to collect detailed morphological trait data for other convergent groups, both to analyse their convergence and to test whether existing methods of measuring convergence can be applied to non-traditional case studies.
%----------------------------------------------------------
\section{Tenrecs}
%Tenrecs are a diverse group
	Tenrecs (Afrosoricida, Tenrecidae) are a morphologically diverse group that is commonly cited as an example of both convergent evolution and adaptive radiation \citep{Soarimalala2011, Eisenberg1969}. % NC: make sure these are the same way round as encountered i nthe intro above
	The Family is comprised of 34 species, 31 of which are endemic to Madagascar \citep{Olson2013}. Body masses of tenrecs span three orders of magnitude (2.5 to > 2,000g); a greater range than all other Families, and most Orders, of living mammals \citep{Olson2003}.
	
%Example of both an adaptive radiation and convergence
	Within this vast size range there are tenrecs which convergently resemble shrews (\textit{Microgale} tenrecs), moles (\textit{Oryzorictes} tenrecs) and hedgehogs (\textit{Echinops} and \textit{Setifer} tenrecs) \citep{Eisenberg1969}. Their similarities include examples of morphological, behavioural and ecological convergence \citep{Soarimalala2011}. Tenrecs are one of only four endemic mammalian clades in Madagascar and the small mammal species they resemble are absent from the island \citep{Garbutt1999}. Therefore, it appears that tenrecs represent an adaptive radiation of species which filled otherwise vacant ecological niches \citep{Soarimalala2011}.
% NC: Be careful with phrasing. You can't "evolve as an adaptive radiation"

%Interesting phylogenetic history
%NC: As it's still about similarity you can probably stick this onto the previous paragraph
	The similarities among tenrecs and other small mammals are even more remarkable when you consider their phylogenetic history. Tenrecs were originally classified within the general "Insectivora" clade and only molecular studies revealed their true phylogenetic affinities within the Afrotherian mammals \citep{Stanhope1998}. Therefore, despite initial appearances, tenrecs are more closely related to elephants, manatees and aardvarks than they are to shrews, moles or hedgehogs. 

%Need a quantitative approach to test these ideas
	Although tenrecs are often cited as an example of an adaptive radiation and of exceptional convergent evolution, these claims have not been investigated quantitatively. There are qualitative similarities between the hind limb morphologies of tenrecs and several other unrelated species with similar locomotory styles \citep{Salton2009} but the degree of morphological similarity has not been established. Morphological diversity is an important feature of adaptive radiations \citep{Losos2010a} and it also informs our understanding of convergent phenotypes \citep{Muschick2012}. Therefore, understanding and quantifying patterns of morphological diversity in tenrecs is of vital importance.
	% NC: I think "vital" importance may be overselling it a bit!
	 My thesis is the first study to address this issue. 

%---------------------------------------------------
\section{Structure \& contents of this thesis}
	
	In this thesis I present the first quantitative study of patterns of morphological diversity in tenrecs. I compiled an extensive data set of morphological characteristics in tenrecs, their closest relatives (golden moles; %NC: Latin name?
	) and the mammals that they convergently resemble. Chapter \ref{chap:methods} includes details about my data collection and the general morphometrics analyses which form the basis for my study. I collected morphological data on the skulls, limbs and skins from 366 specimens representing 99 species of small mammals. 
	
	I used a subset of this data in chapter \ref{chap:disparity} to quantify the morphological diversity of tenrec skulls compared to their closest relatives, the golden moles. The rest of the data represents a significant resource for future studies of morphological diversity and convergence.
	
	Finally, in chapter \ref{chap:discussion}, I discuss the implications of my findings within the context of our understanding of tenrec evolution.% NC: May need to change this depending on what goes into the Discussion :)
 	My results reveal new insights into our understanding of morphological variety in tenrecs and prompt many new questions and possible avenues for further research.


